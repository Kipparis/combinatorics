\section{Permutations}

{\it Addition principle}\index{Addition principle}: if sets $A_{1}$ through $A_{n}$ are pairwise disjoint and have sizes $m_{1},\ldots,m_{n}$, then the size of $A_{1}\cup\ldots\cup A_{n}=\sum\limits_{i=1}^{n}m_{i}$

{\it Multiplication principle}\index{Multiplication principle}: $n$ events have
$m_{i}$ possible outcomes, for $i=1,\ldots,n$, where each $m_{i}$ is unaffected
by the outcomes of other events, then the number of possible outcomes overall is
$\prod\limits_{i=1}^{n}m_{i}$

\section{Additional}

{\it Pigeonhole principle}\index{Pigeonhole principle}: if $n$ items are put
into $m$ containers, with $n > m$, then at least one container must contain more
than one item \cite{pigeonhole-principle}.
